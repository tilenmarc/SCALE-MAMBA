\mainsection{Compiler Pipeline}
\label{sec:compiler}

\subsection{Getting Started}

\subsubsection{Setup}

\paragraph{Dependencies:}

To run the compiler, the following Python packages are required:
\begin{itemize}
\item \texttt{gmpy2} --- for big integer arithmetic
\item \texttt{networkx} --- for graph algorithms
\end{itemize}
If these are not available on your system, you should be able to install them
via \texttt{easy\_install}:

\displaytt{easy\_install --install-dir <path> <package-name>}

You will also need to install \texttt{Rust} if you want to use the
new compiler pipeline, as opposed to the original one.

\paragraph{Directory structure:}

The Compiler package should be located in the same root directory as the
\texttt{compile.sh} script, alongside a folder Programs with sub-folders for
actual programs. The final directory structure should look like:

\begin{lstlisting}[language={}]
root_dir/
|-- compile.sh
|-- Compiler/
|   |-- ...
|   |-- ...
|-- Programs/
|   |-- some_program/
|   |   |-- some_program.mpc
\end{lstlisting}
MAMBA programs have the extension \verb+.mpc+ and each MAMBA
program is placed in its own sub-directory within
the directory \verb+Programs+.

\paragraph{Compilation:}

Compiling a program located in Programs/some\_program/some\_program.mpc is performed by the
command:

\displaytt{compile.sh Programs/some\_program}

\noindent The compiled byte-code and schedule files are then stored in the
same directory as the main program.
Depending on whether you have edited the \verb+compile.sh+ script or
not, this will execute either the new or the old compiler pipeline.
The old pipeline uses the python program \verb+compiler-mamba.py+ to
execute the entire pipeline, indeed this old program can be called independently
with various switches (see below) if you so desire.
The new pipeline uses the same program to generate {\em just} the
assembly, then the assembly is passed to the SCALE-assembler \verb+scasm+ for
processing. The \verb+scasm+ program we believe to be more robust than the
original python program. It is a little slow however, we are working on
speeding this up.

\subsection{The Inner Mamba Compiler}
The core of both pipelines is the program \verb+compile-mamba.py+,
which can itself be called as a standalone program.

\displaytt{compile-mamba.py [options] Programs/some\_program}

\noindent
The options available are as follows

\func{-n --nomerge}
Don't optimize the program by merging independent rounds of communication.

\func{-o --output <name>}
Specify a prefix \verb|name-| for output byte-code files (defaults to the input file name).

\func{-d --debug}
Keep track of the call stack and display when an error occurs.

\func{-c --comparison <type>}
Specify the algorithm used for comparison of secret integers. Can be one of:
\begin{itemize}
\item \verb|log|: logarithmic rounds.
\item \verb|plain|: constant rounds.
\end{itemize}
Default is \verb|log|.

\func{-a --asm-output}
Produces ASM output file for potential debugging use.

\func{-D --dead-code-elimination}
This eliminates instructions which produce an unused result.

\func{-h --help}
Print all available options.

\vspace{5mm}
\noindent
There are a number of other options which are mainly for testing
or developers purposes. These are given by
\begin{description}
	\item \verb|-r --noreorder|
	\item \verb|-M --preserve-mem-order|
	\item \verb|-u --noreallocate|
	\item \verb|-m  -max-parallel-open <MAX_PARALLEL_OPEN>|
	\item \verb|-P --profile|
	\item \verb|-s --stop|
\end{description}
We refer the reader to the compiler help for usage of these compiler options.

\subsubsection{Understanding the compilation output}
The output of the compilation of the inner compiler, assuming it is used to produce byte-code output
and not just machine code, includes important information related to your code.
Although note, much of this information is no longer relevant to later versions of
SCALE, as the system has now evolved beyond what the inner compiler is able to
keep track of.

The main output is an intuitive collection of parameters that can be easily interpreted.
We include a short analysis of the compilation output and a basic description of common
output parameters based on the Simple Example from  Section \ref{sec:example}.
The output is meant to be an informative revision on the different tasks performed by
the compiler. In case you have correctly followed the instructions for compilation,
the output should resemble the following:

\begin{verbatim}
Compiling program in Programs/tutorial
tutorial
p = 34359738421
Prime size: 32
Default bit length: 24
Default statistical security parameter: 6
Compiling file Programs/tutorial/tutorial.mpc
Compiling basic block tutorial-0--0
Compiling basic block tutorial-0-begin-loop-1
Compiling basic block tutorial-0-end-loop-2
Compiling basic block tutorial-0-if-block-3
Compiling basic block tutorial-0-else-block-4
Compiling basic block tutorial-0-end-if-5
Processing tape tutorial-0 with 6 blocks
Processing basic block tutorial-0--0, 0/6, 4386 instructions
Program requires 7 rounds of communication
Program requires 440 invocations
Processing basic block tutorial-0-begin-loop-1, 1/6, 21 instructions
WARNING: Order of memory instructions not preserved, errors possible
Program requires 1 rounds of communication
Program requires 1 invocations
Processing basic block tutorial-0-end-loop-2, 2/6, 18 instructions
Program requires 1 rounds of communication
Program requires 1 invocations
Processing basic block tutorial-0-if-block-3, 3/6, 2 instructions
Processing basic block tutorial-0-else-block-4, 4/6, 2 instructions
Processing basic block tutorial-0-end-if-5, 5/6, 14 instructions
Program requires 1 rounds of communication
Program requires 1 invocations
Tape register usage: defaultdict
(<function <lambda> at 0xf27c80>, {'c': 1278, 'ci': 10, 's': 3793})
modp: 1278 clear, 3793 secret
Re-allocating...
Register(s) [ci0] never used, assigned by 'popint ci0' in <omitted>
Compile offline data requirements...
Tape requires 211 triples in modp, 100 squares in modp, 184 bits in modp
Tape requires prime bit length 32
Program requires:
{('modp', 'triple'): 211, ('modp', 'square'): 100, ('modp', 'bit'): 184}
Cost: 0.102230483271
Memory size: defaultdict(<function <lambda> at 0xf27b18>,
{'c': 8192, 'ci': 8192, 's': 8292})
Compiling basic block tutorial-0-memory-usage-6
Writing to Programs/tutorial/tutorial.sch
Writing to Programs/tutorial/tutorial-0.bc
Writing to Programs/tutorial/tutorial-0.bc
\end{verbatim}
The compilation output in this case corresponds  to a 3-party scenario using Shamir Secret Sharing for both, online and offline phases.
The output introduces first some program level parameters, followed by the rolling of the tape (converting all operations to byte-code instructions), and offline data requirements. We now proceed to analyze the output more into detail.

\subsubsection{Program Level Parameters}

\func{p}
This is the modulus of the finite field which secret sharing is defined over.
It is defined through \verb|Setup.x|.
The program \verb+Setup.x+ then stores it in the \verb|Data| folder, more specifically in the file \verb|SharingData.txt|

\func{Prime Size}
Is the bit size of the prime $p$ above.

\func{Default Bit Length}
Is the default maximum bit length of emulated integer values inputs,
for operations like integer comparison below.
In particular see the value $k$ below in Section \ref{ref:datatypes}.
Because of mechanisms such as comparisons, implemented under statistical security parameters, the input size has to be adjusted so
some power of 2 smaller than the modulus.

\func{Default Statistical Security}
By default, some bits are reserved for the statistical security of operations such as comparisons. This reduces the actual input size.
The field size has to be greater than the inputs bit-length $k$ plus the statistical security bit-length $\kappa$ such that no overflow occurs.
When the prime $p$ is 128-bits in length the default values of $\kappa=40$ and $k=64$ are chosen.
These can be altered by using the commands
\begin{lstlisting}[language={python}]
program.security = 100
program.bit_length = 20
\end{lstlisting}
Remember, the requirement is that $k+\kappa$ must be less than the bit length of the prime $p$.
These are the parameters for the base \verb+sint+ data type when we interpret the value it holds
as an integer, as opposed to an element modulo $p$.
There are also $\kappa$ statistical security parameters associated with the \verb+sfix+ and \verb+sfloat+ data types;
whose default values are also $40$.
These can be set by
\begin{lstlisting}[language={python}]
sfix.kappa=50
sfloat.kappa=50
\end{lstlisting}


\subsubsection{Compilation comments regarding the tape enrollment:}
The compiler output showcases a typical example of instructions from the compiler. They reflect the tasks being performed while writing down the instructions in the byte-code.
\func{Compiling basic block}
Signals the start of compilation of a new basic block on the tape. It also adds operations to such block.
\func{Processing tape}
Signals the start of the optimization of the contents of the tape. This includes merging blocks to reduce rounds. It also eliminates dead code.
\func{Processing basic block}
While Processing tape, blocks get optimized, code reviewed to eliminated dead code and also merged.
\func{Program requires X rounds of communication}
Total amount of communication rounds (latency) of the program after compilation and its optimization.
\func{Program requires X invocations}
Total amount of multiplications needed for the compiled program (amount of work) to process secret data.

\subsubsection{Offline data Requirements:}

\func{Tape requires X triples in $\modp$, Y squares in $\modp$, Z bits in $\modp$}
Signal a recount of the amount of different triples needed for the protocol execution.
However, this is only accurate if you are using no advanced pre-processing such
as daBits. In addition the daBit production will require itself
pre-processing of triples, even if you do not use them. Thus this is purely an estimate
for ``legacy'' MAMBA programs which do not use the new functionality, and
does not accurately measure how much pre-processing is needed by the
entire system.
It also does not meaure the amount of OT-extensions needed to produced
authenticated-ANDs for the garbling process.
So please take these numbers with a {\em huge} pinch of salt.

\func{Memory Size}
This is an estimation of the memory allocation needed to execute the protocol.

\subsection{The Old Compilation Pipeline}
Compilation for the old compilation pipeline is performed by a single call to the
Python function \verb|compile-mamba.py| on the main source code file. The creation and optimization
of byte-code tapes is done on-the-fly, as the program is being parsed. As soon
as a tape is finished parsing it is optimized and written to a byte-code
file, and all its resources freed to save on memory usage.
This can be executed with the default flags etc by calling

\displaytt{./compile-old.sh target}

\noindent or, if you edit \verb+compile.sh+ to point to the old compilation
pipeline...

\displaytt{./compile.sh target}

During parsing, instructions are created such that every output of an
instruction is a new register. This ensures that registers are only written to
once, which simplifies the register allocation procedure later.
The main goal of the optimization process is to minimize the number of rounds
of communication within basic blocks of each tape, by merging independent
\verb|startopen| and \verb|stopopen| instructions. This is done through
instruction reordering and analysis of the program dependency graph in
each basic block. After optimization, a simple register allocation procedure
is carried out to minimize register usage in the virtual machine.

\subsubsection{Program Representation}

The data structures used to represent a program are implemented in
\verb|program.py|, and here the relevant classes are described.

\begin{class}{Program}

The main class for a program being compiled.
\begin{description}
\item[tapes:] List of tapes in the program.
\item[schedule:] The schedule of the tapes is a list of pairs of the form
\verb+(`start|stop', list_of_tapes)+. Each pair instructs the virtual machine to
either start or stop running the given list of tapes.
\item[curr_tape:] A reference to the tape that is currently being processed.
\item[curr_block:] A reference to the basic block that is currently being
processed (within the current tape). Any new instructions created during
parsing are added to this block.
\item[restart_main_thread():] Force the current tape to end and restart a fresh
tape. Can be useful for breaking tapes up to speed up the optimization process.
\end{description}
\end{class}

\begin{class}{Tape}
Contains basic blocks and data for a tape. Each tape has its own set of
registers
\begin{description}
\item[basicblocks:] List of basic blocks in this tape.
\item[optimize():] Optimize the tape. First optimizes communication, then
inserts any branching instructions (which would otherwise interfere with the
previous optimization) and finally allocates registers.
\item[req_num:] Dictionary storing the required numbers of preprocessed
triples, bits etc. for the tape.
\item[purge():] Clear all data stored by the tape to reduce memory usage.
\item[new_reg(reg_type, size=None):] Creates a register of type \verb|reg_type|
(either `s' or `c' for secret or clear)
for this Tape. If \verb|size| is specified and $> 1$ then a vector of registers
is returned.
\end{description}
\end{class}

\begin{class}{BasicBlock}
A basic block contains a list of instructions with no branches, but may
end with a branch to another block depending on a certain condition. Basic
blocks within the same tape share the same set of registers.
\begin{description}
\item[instructions:] The instructions in this basic block.
\item[set_exit(condition, exit_block):] Sets the exit block to which control
flow will continue, if the instruction \verb|condition| evaluates to true.
If \verb|condition| returns false
or \verb|set_exit| is not called, control flow implicitly
proceeds to the next basic block in the parent Tape's list.
\item[add_jump():] Appends the exit condition instruction onto the list of
instructions. This must be done \emph{after} optimization, otherwise instruction
reordering during merging of opens will cause the branch to behave incorrectly.
\item[adjust_jump():] Calculates and sets the value of the relative jump
offset for the exit
condition instruction. This must be done after \verb|add_jump| has been called
on \emph{every basic block in the tape}, in order that the correct jump offset
is calculated.
\end{description}
\end{class}

\begin{class}{Tape.Register}
The class for clear and secret registers. This is enclosed within a Tape, as
registers should only be created by calls to a Tape's \verb|new_reg| method
(or, preferably, the high-level library functions).
\end{class}

\begin{class}{Instruction}
This is the base class for all instructions. The field \verb|__slots__| should
be specified on every derived class to speed up and reduce the memory usage from
processing large quantities of instructions. Creating new instructions should
be fairly simple by looking at the existing code. In most cases,
\verb|__init__| will not need to be overridden; if this is necessary, ensure
that the original method is still called using \verb|super|.

\begin{description}
\item[code:] The integer opcode. This should be stored in the \verb|opcodes|
dictionary.
\item[arg_format:] The format for the instruction's arguments is given as a
list of strings taking one of the following:
\begin{itemize}
\item `c[w]': clear register, with the optional suffix `w' if the register is
written to.
\item `s[w]': secret register, as above.
\item `r[w]': regint register, as above.
\item `i'   : 32-bit integer signed immediate value.
\item `int' : 64-bit integer unsigned immediate value.
\item `p'   : 32-bit number representing a player index.
\item `str' : A four byte string.
\end{itemize}
\end{description}

For example, to implement a basic addition instruction for adding a secret
and a clear register, the following code suffices.

\begin{lstlisting}
class addm(Instruction):
    """ Add a secret and clear register into a secret register. """
    __slots__ = [] # contents inherited from parent
    code = opcodes['ADDM']
    arg_format = ['sw', 's', 'c']
\end{lstlisting}

\end{class}

\begin{class}{Memory}
Memory comes in three varieties \verb+sint+, \verb+cint+, and
\verb+regint+; denoted by \verb+S[i]+, \verb+C[i]+ and \verb+R[i]+.
\end{class}

\subsubsection{Optimizing Communication}

The first observation is that communication of small data items
costs, so we want to pack as much data together in a start/stop open
command.
The technique for automating this is as follows:
\begin{itemize}
  \item Calculate the \textit{program dependence graph} $G$, whose vertices correspond to
        instructions in the byte-code;
        treating start/stop open instructions as a single instruction.
        Two vertices $(v_i,v_j)$ are connected by a directed edge if an output from
        instruction $v_i$ is an input to instruction $v_j$.
  \item Now consider the graph $H$, whose vertices also correspond to
    instructions; insert a directed edge into $H$ from every
    \verb+startopen+ vertex to any other vertex where there is a path
    in $G$ not going through another \verb+startopen+ vertex.
  \item Label vertices in $H$ with their {\em maximal} distance from
    any source in $H$ that is a \verb+startopen+.
  \item Merge all start/stop opens with the same label in $H$.
  \item Create another graph $H'$ with vertices corresponding to
    instructions and edges from every non-open vertex to any
    \verb+startopen+ where there is a path not going though another
    \verb+startopen+ vertex.
  \item Label non-open vertices in $H'$ with the minimum of the labels
    (in $H$) of their successors in $H'$.
  \item Compute a topological ordering the merged graph such that
    sources with $H'$-label $i$ appear after the open with label $i-1$
    and such that sinks with $H$-label $j$ before the open with label
    $j$. Furthermore, let non-open vertices with $H$-label $i$ and
    $H'$-label $j$ such that $j - i \ge 2$ appear between the
    \verb+startopen+ and \verb+stopopen+ with label $i+1$.
  \item Output the resulting instruction sequence.
\end{itemize}

\subsubsection{Register allocation}
\label{sec:regalloc}

After reordering instructions to reduce communication costs, we are left with
a sequence of instructions where every register is written to no more than
once. To reduce the number of registers and memory requirements we now
perform register allocation.

Traditionally, this is done by graph colouring, which requires constructing
the \emph{interference
graph}, where nodes correspond to registers, with an undirected edge $(a,b)$ if
the `lifetimes' of registers $a$ and $b$ overlap. Algorithms for creating this
graph, however, typically run in $O(n^2)$ time for $n$ instructions. Since we
unroll loops and restrict the amount of branching that can be done, basic blocks
are a lot larger than usual, so constructing the interference graph becomes
impractical. 
We instead use a simple method that takes advantage of the fact that 
the input program is in SSA form. We iterate backwards over the instructions,
and whenever a variable is assigned to, we know that the variable will not be
used again and the corresponding register can be re-used.

Note that register allocation should be done \emph{after} the merging of open
instructions: if done first, this will severely limit the possibilities for
instruction reordering due to reuse of registers.


\subsubsection{Notes}
The following are mainly notes for the development team, so we
do not forget anything. Please add stuff here when you notice
something which might be useful for people down the line.

\begin{enumerate}
\item Instructions/byte-codes can be re-ordered. Sometimes this is a bad idea,
e.g. for IO instructions. All instructions/byte-codes which inherit from
\verb+IOInstruction+ are never reordered.
\item Instructions executed in the test scripts need to be emulated in
python. So do not use an instruction here which does not have an emulation.
The emulation stuff is in \verb+core.py+.
Essentially, there's two options for testing:A
\begin{itemize}
\item If you use the 'test(x,y)' syntax you have to make sure that all functions (including classes) called for x are defined in \verb+core.py+, but they don't have to do anything. For example:
\begin{verbatim}
       \verb+ print_ln = lambda *args: None+
\end{verbatim}
This allows to call print_ln in tested programs, but nothing really happens when testing the results.
\item If you use the 'test(x)' syntax you have to make sure that the functionality is replicated in \verb+core.py+ and understood by \verb+Scripts/test_results.py+. This is more involved, but it allows to write tests with less effort.
\end{itemize}
\end{enumerate}

\subsection{The New Compilation Pipeline}
\label{sec:newcompiler}
The new compilation pipeline works by calling the inner compiler with the flags

\displaytt{compile-mamba.py -A -n -r -M -D -u -s Programs/some\_program}

\noindent The \verb|-A| option produces the output \verb+.asm+ files in the directory
\verb+Programs/some\_program+.
To compile these \verb+.asm+ files the assembler \verb+scasm+ is called

\displaytt{./scasm --verbose Programs/some\_program}

\noindent Which is itself an alias for the \verb+Rust+ program in the directory \verb+Assembler+.
The entire new compilation pipeline can be executed with the default 
flags etc by calling

\displaytt{./compile-new.sh target}

\noindent or, if you edit \verb+compile.sh+ to point to the old compilation
pipeline...

\displaytt{./compile.sh target}

\noindent If you want to pass various optimization flags then you
execute

\displaytt{./compile.sh -O2 target}

\noindent The default optimization level is \verb+-O3+.
The meaning of the flags is as follows, the effect of
the optimizations is cumulative,
\begin{itemize}
\item[-O0:] Apply no optimizations to the assembler, simply
output the correspond byte-codes.
\item[-O1:] Apply register painting so as to reduce the number
of registers required.
\item[-O2:] Merge \verb+startopen+ and \verb+stopopen+ commands
as described below.
\item[-O3:] Try to insert more instructions between a
\verb+startopen+ and \verb+stopopen+.
\end{itemize}
Note, that for very large \verb+.asm+ files, the last
two optimizations can take a long time.

\subsubsection{scasm Commands}
As a \verb+Rust+ program we can control the use of \verb+scasm+ with more fine-grain than
just using the command line tool above.
We give some examples/notes here of different call patterns plus our testing scripts etc.
To make binaries and run it on a single \verb+.asm+ file execute, within
the \verb+Assembler+ directory, ...

\displaytt{cargo run --bin scasm somefile.asm}

\noindent The binary has a bunch of flags to change what kind of output you want, they are documented under

\displaytt{cargo run --bin scasm -- --help}

\noindent If you want to go from an \verb+.asm+ file to a \verb+.bc+ file, all you need is

\displaytt{cargo run --bin scasm -- input_file.asm output_file.bc}

\noindent The opposite direction is possible, too, so you can use \verb+scasm+ as a disassembler if you want
to inspect binary files.

\noindent The addition of a \verb+--release+ flag will compile \verb+scasm+ with optimizations so it's much faster.
The optimizations which \verb+scasm+ runs on the assembly files are not affected by the \verb+--release+ flag.
To turn on the debug output you can set the

\displaytt{export RUST_LOG=scasm::transforms::optimizer_step1}

\noindent environment flag to turn on all of the debug printing in the
\verb+transforms::optimizer_step1+ module.  The same goes for any other module.
To only print out debug info use...

\displaytt{export RUST_LOG=scasm::transforms::optimizer_step1=debug}

\noindent The assembler also produces \verb+.dot+ file to visualize the 
blocks etc using package such as Graphviz. 
To obtain \verb+.dot+ output, specify the output file with a \verb+.dot+ extension.

\subsubsection{scasm Shell Program}
In the main directory you will find the scasm shell program \verb+scasm+,
using this is the recommended (non-developer) way of calling \verb+scasm+.
To see how this operates look at the commented out commands in
\verb+compile-new.sh+.
If you just want to compile all \verb+.asm+ files in directory
\verb+target+ then execute

\displaytt{./scasm target}

\noindent
For example the command, if you want to execute \verb+scasm+ with 
only the optimization related to merging \verb+startopen+ and 
\verb+stopopen+ executed, i.e. no register coloring is performed,
plus also see possible warnings in the code, then execute the command, 

\displaytt{./scasm --verbose target -- --optimizations start_stop_open}

\noindent With \verb+verbose+ you get a lot of warnings of instructions
which write but do not read from a register. These are {\em almost
always} perfectly acceptable as they can represent instructions
which write to multiple registers of which only one ends up
being used later on.

To make the assembler output \verb+.asm+ files at the end
of each optimization step (this is useful to find problems in the
output of the assembler) use the command

\displaytt{./scasm  target -- --dump-optimizations=temp}

\noindent
Various optimizations can be turned on and off using the 
\verb+-O0, -O1, -O2, -O3+ flags. Again the default is to execute
\verb+-O3+.

\displaytt{./scasm  target -- -O2}

\subsubsection{scasm Testing}
To run all tests in the \verb+tests+ directory (which are just a light form of testing)

\displaytt{cargo test}

\noindent again in the \verb+Assembler+ directory.
If something fails this will produce files of the form \verb+.stderr+.
If you then make edits, for example to \verb+instructions.rs+, or the parser etc to remove the
errors then the errors will disappear.

\vspace{5mm}

\noindent To run all tests in the \verb+scasm-tests+ directory, which is our standard way of testing
for major bugs within the assembler, execute the following steps:
\begin{enumerate}
\item In the main SCALE directory run

\displaytt{./Scripts/compile-scasm}
This creates the asm files in the \verb+scasm-tests+ directory
\item To run all tests execute

\displaytt{cargo run --bin scasm_tests --release}
\end{enumerate}

Note that these commands will take quite some time as the test are multiple hundred megabytes in size.

\subsubsection{scasm Internals}
The \verb+scasm+ assembler work flow is organized as follow, with the main entry point
being \verb+main.rs+.
\begin{itemize}
\item Read input
\item Parse input
\item Optimize the assembly
\item Print out assembly (the relexer)
\item and/or output the byte-code.
\end{itemize}
Internally a \verb+Body+ is a file, a \verb+Body+ is split into
\verb+Blocks+, with each \verb+Block+ terminated by a \verb+jmp+-like instruction.
Finally a \verb+Body+ contains \verb+Statements+.

\vspace{5mm}

\noindent To add new instructions you need to
\begin{enumerate}
\item Add them to the parser to read stuff in
\item Add them to the relexer to get stuff out
      In the relexer note that an instruction needs two arguments,
      even if none are present. See for example the VMControl
      instructions

      \displaytt{Instruction::VMControl(instr) => (instr, vec![])}
\item Add the instructions to the file \verb+Assembler/src/binary/instructions.rs+.
      This contains various flags which detail how an instruction is used
      (read/write/memory and channel dependencies), it is also used to generate
      the documentation.
\end{enumerate}
To compile the \verb+instructions.rs+ file into LaTeX to include in
this manual use

\displaytt{cargo test generate_instruction_tex_table}

\noindent Although when you run \verb+cargo test+ this is automatically
generated in any case. \\

\noindent
Finally when editing \verb+scasm+ remember to execute

\displaytt{cargo fmt}

\noindent before committing. If you are running visual studio code inside the \verb+Assembler+ directory
the formatting will happen automatically as you type.

\subsubsection{scasm Optimization}
We make the following assumptions of an input file to the assembler:
\begin{itemize}
\item An input assembler file has basic blocks which are in SSA (Static Single Assignment) form, 
which means each register is assigned to only once within the basic block.
And in addition a register is only assigned to in one instruction in the entire
file (although the instruction may be executed multiple times due to loops).
Thus on input the number of registers have not been reduced, this
happens during our optimization stages below.

\item We assume the input has no read-before-writes on registers, since these give
undefined behaviour. Although \verb+scasm+ will emit an error message if this is not upheld.

\item A comment line is indicated by prepending with a $\#$.

\item Arguments to instructions are separated by a comma, e.g. \verb+ldmc c1, 10+.

\item Input of instructions can be in either upper or lower case, we do not care what you want to use in this
regard. However, all registers are in lower case.

\item Registers are denoted by a type followed by a number, where the types can be
one of \verb+c+, \verb+s+, \verb+r+, \verb+sr+, and \verb+sb+. These
correspond to \verb+cint+, \verb+sint+, \verb+regint+, \verb+sregint+ and
\verb+sbit+.

\item On input a \verb+startopen+ is always followed by the corresponding
\verb+stopopen+.
\end{itemize}

The main optimization is to merge \verb+startopen+/\verb+stopopen+ instructions;
this is executed in optimization level \verb+-O2+ or \verb+-O3+.
We take a {\em very} conservative approach here, we guarantee that no memory
access are violated, and that the behaviour with respect to interactions with any
external system is not violated.
In particular this means that placement of barrier instructions, marked with a
$\dagger$ above are respected absolutely.
This ensures (for example) IO and debug-printing happen not only in the correct
order, but also where the programmer expects them to appear.
We take a basic block of instructions, recall these end in a jmp type instruction,...
\begin{verbatim}
         ldmc c1, 10
         ldms s8, 20
         addm s1,s2,c1
         ldmc c2, 30
         addc c3,c1,c2
         adds s3,s1,s2
         startopen 1, s3
         stopopen 1, c4
         adds s4, s1, s2
         addm s5, s1, c4
         startopen 1, s5
         stopopen 1, c6
         stmc c6, 50
         addm s9, s2, c2
         startopen 1, s9
         stopopen 1, c5
         ldms s7, 110
         startopen 1, s7
         stopopen 1, c9
         print_reg c5
         startopen 1, s8
         stopopen 1, c10
         addc c7, c6, c10
         ldmc c8, 30
         mulm s6, s2, c8
         JMP if xxxx to LL
\end{verbatim}
Note that this basic block can start with registers which are already allocated,
e.g. register \verb+s2+ in the above snippet, but we must be in SSA form (no register
can be written to more than once).


\paragraph{Step 1}
To process a basic block of instructions we proceed as follows; note
that some registers may have been defined prior to the execution
of this basic block.
To each instruction we first assign an instruction and round depth,
and to each register we also assign an instruction and round depth.
Let us call these $i_d$, $i_r$, $r_d$ and $r_r$.
This is done as follows:
\begin{itemize}
  \item We keep two variables \verb+mem_i_depth+ and \verb+mem_r_depth+
         and assign them to $-1$ and $0$. These are essentially the `instruction depth'
         and `round depth' of a special register called `memory'.
  \item We keep a variable called \verb+block_num+ which is initially set to zero.
  \item We assign all registers to have initial instruction and round depth $-1$
        as well, so we set $r_d=r_r=-1$ for all registers $r$.
  \item When we meet a new instruction we compute $m_d = \max r_d$ and
         $m_r = \max r_r$ for all {\em read} registers $r$. If there are no
         read registers we set $m_d=-1$ and $m_r=0$.
  \item When we meet an instruction which just reads from a register
        we assign $i_d$ and $i_r$ the value of $m_d+1$ and $m_r$.
        [This includes a \verb+startopen+ operation].
  \item When we meet an instruction which writes to a register
    we assign $i_d$ and $r_d$ (for the written register $r$) the value of $m_d+1$.
    We assign $i_r$ and $r_r$ the value of $m_r$.
    If the write instruction register before this assignment has a depth
    already assigned then we should abort (as this is not correct input assembly, which should
    be in SSA form).
  \item When we meet a load memory instruction we assign (akin to when
    we read a register) we set $i_d$ to
    \[ \max(\verb+mem_i_depth+, m_d)+1, \]
    and we set $i_r$ to
    \[ \max(\verb+mem_r_depth+, m_r). \]
    We assign $r_d$ the value of $i_d$ and $r_r$ the value of $i_r$.
  \item When we meet a store memory instruction we set \verb+mem_i_depth+ to be
    the
    \[ \max(\verb+mem_i_depth+, m_d)+1. \]
    We then set $i_d$ to be \verb+mem_i_depth+, and $i_r$ and \verb+mem_r_depth+
    to be
    \[ \max(\verb+mem_r_depth+, m_r). \]
  \item For a \verb+stopopen+ [which before optimization always follows a \verb+startopen+]
    we take the $i_d$ and $i_r$ of the previous \verb+startopen+ and then
    assign the new $i_d$ and new $i_r$ as one more than these values.
    The associated values of $r_d$ and $r_r$, for the written registers, are
    assigned the same values as well.
  \item For any other instruction we assign the instruction and round depth to zero.
  \item Each instruction gets assigned the current value of block number.
  \item When we meet a barrier instruction we increase the block number by one.
  \item Peek and GetSp operations are considered as memory read instructions, whereas
        Poke and Push operations are considered as memory write instructions in the above
        methodology.
        Pop on the other hand both reads memory (takes something off the stack),
        and writes to memory (alters the stack itself).
        Of course Peek, Pop and GetSp are also register write instructions and
        Poke and Push are register read instructions.
  \item The GarbledCircuit and LocalFunction operations are considered as both
        memory read and memory write operations, as they both push and pop from
        the stack.
  \item Note when doing {\em all} of the above we have to also remember that vectorized
        instructions touch more than just the instruction mentioned in the opcode.
\end{itemize}
Once we have done this we obtain the following values for the instruction and
round depths' of the instructions in our basic block...
\begin{verbatim}
                   instr_depth   rd_depth  mem_i_dpt  mem_r_dpt
  ldmc c1, 10           0            0        -1         0
  ldms s8, 20           0            0        -1         0
  addm s1,s2,c1         1            0        -1         0
  ldmc c2, 30           0            0        -1         0
  addc c3,c1,c2         1            0        -1         0
  adds s3,s1,s2         2            0        -1         0
  startopen 1, s3       3            0        -1         0
  stopopen 1, c4        4            1        -1         0
  adds s4, s1, s2       2            0        -1         0
  addm s5, s1, c4       5            1        -1         0
  startopen 1, s5       6            1        -1         0
  stopopen 1, c6        7            2        -1         0
  stmc c6, 50           8            2         8         2
  addm s9, s2, c2       1            0         8         2
  startopen 1, s9       2            0         8         2
  stopopen 1, c5        3            1         8         2
  ldms s7, 110          9            2         8         2
  startopen 1, s7      10            2         8         2
  stopopen 1, c9       11            3         8         2
  print_reg c5          4            1         8         2   <- Barrier instruction
  startopen 1, s8       1            0         8         2
  stopopen 1, c10       2            1         8         2
  addc c7, c6, c10      8            2         8         2
  ldmc c8, 30           9            2         8         2
  mulm s6, s2, c8      10            2         8         2
  JMP if xxxx to LL
\end{verbatim}
We obtain the associated register depths as
\begin{verbatim}
        c1, c2, c3, c4, c5, c6, c7, c8, c9, c10, s1, s2, s3, s4, s5, s6, s7, s8, s9
instr    0   0   1   4   3   7   8   9  11    2   1  -1   2   2   5  10   9   0   1
round    0   0   0   1   1   2   2   2   3    1   0  -1   0   0   1   2   2   0   0
\end{verbatim}

\paragraph{Step 2}
We now merge the \verb+startopen+ and \verb+stopopen+ commands
which have the same round depth and the same block number.
This means we do not merge instructions which are
separated by a barrier command.
\begin{verbatim}
                   instr_depth   rd_depth  mem_i_dpt  mem_r_dpt
  ldmc c1, 10           0            0        -1         0
  ldms s8, 20           0            0        -1         0
  addm s1,s2,c1         1            0        -1         0
  ldmc c2, 30           0            0        -1         0
  addc c3,c1,c2         1            0        -1         0
  adds s3,s1,s2         2            0        -1         0
  startopen 2, s3, s9   3            0        -1         0
  stopopen 2, c4, c5    3            1        -1         0
  adds s4, s1, s2       2            0        -1         0
  addm s5, s1, c4       5            1        -1         0
  startopen 1, s5       6            1        -1         0
  stopopen 1, c6        7            2        -1         0
  stmc c6, 50           8            2         8         2
  addm s9, s2, c2       1            0         8         2
  ldms s7, 110          9            2         8         2
  startopen 1, s7      10            2         8         2
  stopopen 1, c9       11            3         8         2
  print_reg c5          4            1         8         2   <- Barrier instruction
  startopen 1, s8       1            0         8         2
  stopopen 1, c10       2            1         8         2
  addc c7, c6, c10      8            2         8         2
  ldmc c8, 30           9            2         8         2
  mulm s6, s2, c8      10            2         8         2
  JMP if xxxx to LL
\end{verbatim}
The problem now is that the instruction depths will be wrong,
and instructions may not be in a write-before-read order.
For example \verb+s9+ above is opened before it is written to.

\paragraph{Step 3}
Noting which registers were defined on entry (i.e. registers
with $r_d=-1$) we now run the same algorithm again.
We have to do multiple passes through the instructions until all
instructions are assigned an instruction depth. We need
to know the registers defined on entry, otherwise this
we dont know whether an instruction is ok-as-is or needs
to be defined later.
\begin{verbatim}
                   instr_depth   rd_depth  mem_i_dpt  mem_r_dpt
  ldmc c1, 10           0            0        -1         0
  ldms s8, 20           0            0        -1         0
  addm s1,s2,c1         1            0        -1         0
  ldmc c2, 30           0            0        -1         0
  addc c3,c1,c2         1            0        -1         0
  adds s3,s1,s2         2            0        -1         0
  startopen 2, s3, s9   3            0        -1         0     <-
  stopopen 2, c4, c5    4            1        -1         0     <-
  adds s4, s1, s2       2            0        -1         0
  addm s5, s1, c4       5            1        -1         0     <-
  startopen 1, s5       6            1        -1         0     <-
  stopopen 1, c6        7            2        -1         0     <-
  stmc c6, 50           8            2         8         2     <-
  addm s9, s2, c2       1            0        -1         0
  ldms s7, 110          9            2         8         2     <-
  startopen 1, s7      10            2         8         2     <-
  stopopen 1, c9       11            3         8         2     <-
  print_reg c5          4            1         8         2     <- Barrier
  startopen 1, s8       1            0        -1         0
  stopopen 1, c10       2            1        -1         0
  addc c7, c6, c10      8            2         8         2     <-
  ldmc c8, 30           9            2         8         2     <-
  mulm s6, s2, c8      10            2         8         2     <-
  JMP if xxxx to LL
\end{verbatim}
Instructions marked with a \verb+<-+ are defined on the second pass,
through the list. Note the \verb+stmc+ means we cannot process any future
\verb+ldmc+ on the first pass.
We obtain the associated register depths as
\begin{verbatim}
        c1, c2, c3, c4, c5, c6, c7, c8, c9, c10, s1, s2, s3, s4, s5, s6, s7, s8, s9
instr    0   0   1   4   4   7   8   9  11    2   1  -1   2   2   5  10   9   0   1
round    0   0   0   1   1   2   2   2   3    1   0  -1   0   0   1   2   2   0   0
\end{verbatim}


\paragraph{Step 4}
We now sort with respect to the following variables (in order
of priority)
(block number, round depth, instruction depth).
We do not alter the position of any barrier operations,
thus barrier operations still marks the change between
a given instruction block and the next one.
This gives us...
\begin{verbatim}
                   instr_depth   rd_depth  mem_i_dpt  mem_r_dpt
  ldmc c1, 10           0            0        -1         0
  ldms s8, 20           0            0        -1         0
  ldmc c2, 30           0            0        -1         0
  addm s1,s2,c1         1            0        -1         0
  addc c3,c1,c2         1            0        -1         0
  addm s9, s2, c2       1            0        -1         0
  adds s3,s1,s2         2            0        -1         0
  adds s4, s1, s2       2            0        -1         0
  startopen 2, s3, s9   3            0        -1         0     <-
  stopopen 2, c4, c5    4            1        -1         0     <-
  addm s5, s1, c4       5            1        -1         0     <-
  startopen 1, s5       6            1        -1         0     <-
  stopopen 1, c6        7            2        -1         0     <-
  stmc c6, 50           8            2         8         2     <-
  ldms s7, 110          9            2         8         2     <-
  startopen 1, s7      10            2         8         2     <-
  stopopen 1, c9       11            3         8         2     <-
  print_reg c5          4            1         8         2     <- Barrier
  startopen 1, s8       1            0        -1         0
  stopopen 1, c10       2            1        -1         0
  addc c7, c6, c10      8            2         8         2     <-
  ldmc c8, 30           9            2         8         2     <-
  mulm s6, s2, c8      10            2         8         2     <-
  JMP if xxxx to LL
\end{verbatim}


\subparagraph{Step 4a}
\label{sect4a}
To cope with the following case of vectorized start/stop opens
we increase the variable \verb+block_num+ whenever a different start/stop
open vectorization is encountered. This leads to slightly
less merging...
\begin{verbatim}
        vstartopen 10, 1, s0        0
        vstopopen 10, 1, c0         0
        startopen 1, s101           1
        stopopen  1, c101           1
        startopen 1, s100           1
        stopopen  1, c100           1
        vstartopen 10, 1, s30       2
        vstopopen 10, 1, c30        2
\end{verbatim}
Thus we get the merged instructions
\begin{verbatim}
        vstartopen 10, 1, s0        0
        vstopopen 10, 1, c0         0
        startopen 1, s101, s100     1
        stopopen  1, c101, c100     1
        vstartopen 10, 1, s30       2
        vstopopen 10, 1, c30        2
\end{verbatim}
But the two vectorized start and stops do not get merged.


\paragraph{Step 5}
If optimization level is set to \verb+-O3+ we now work out which 
additional instructions we can place between
a \verb+startopen+ and a \verb+stopopen+.
Note, the previous step can result in instructions being
inserted between \verb+startopen+ and a \verb+stopopen+
commands, but there may be additional ones which this
step tries to find.
To do this we execute the following steps
\begin{enumerate}
\item We take all instructions with a given round depth and given
block number. This set of instructions contains at most one
\verb+startopen+ instruction due to the discussion in Section \ref{sect4a}.
The set of {\em interesting} instructions is the set of instructions
of a given round depth and block number which occur {\em before}
the \verb+startopen+. If there is no \verb+startopen+ then there
are no interesting instructions for this round depth and block number.
\item We let $R$ denote the set of all registers in the given
\verb+startopen+.
\item \label{it3} We now go through the set of interesting instructions
and searching for instructions which write to a register
in the set $R$. We ``delete'' these instruction from our set of
interesting instructions, and add the associated read registers from this
instruction into the set $R$.
Note, we again treat `memory' here as a special register for this discussion.
\item We repeat the line \ref{it3} until no more instructions
are ``deleted'' from the set of interesting instructions.
\item The remaining interesting instructions are then moved to
a position directly after the given \verb+startopen+.
\end{enumerate}
This gives us...
\begin{verbatim}
  ldmc c1, 10
  ldmc c2, 30
  addm s1,s2,c1
  addm s9, s2, c2
  adds s3,s1,s2
  startopen 2, s3, s9
  ldms s8, 20
  addc c3,c1,c2
  adds s4, s1, s2
  stopopen 2, c4, c5
  addm s5, s1, c4
  startopen 1, s5
  stopopen 1, c6
  stmc c6, 50
  ldms s7, 110
  startopen 1, s7
  stopopen 1, c9
  print_reg c5
  startopen 1, s8
  stopopen 1, c10
  addc c7, c6, c10
  ldmc c8, 30
  mulm s6, s2, c8
  JMP if xxxx to LL
\end{verbatim}

\subsubsection{Things to have in mind}
Sometimes the compiler will produce no output to be able to run SCALE on your
new \verb+mpc+ program. More often is that the instructions provided in the
program invalidate some constraint. Take for example:

\begin{center}
\begin{tabular}{|l|l|} \hline
\textbf{Incorrect}                        & \textbf{Correct}                       \\ \hline
x = cint(0)                              & x = cint(0) \\
    &   result = MemValue(cint(x)) \\
a = cint(3)                             & a = cint(3) \\
if_then(a != 0) & if_then(a != 0) \\
x += 200 & result.write(x + 200) \\
else_then() & else_then() \\
x += 100 & result.write(x + 100) \\
end_if() & end_if() \\
    & x = result.read() \\
print_ln('\%s', x) & print_ln('\%s', x) \\ \hline
\end{tabular}
\begin{footnotesize}
\\ MAMBA block usage
\end{footnotesize}
\end{center}
This should normally output $200$ in both cases but in the 
left case the MAMBA compiler outputs no assembler for \verb+scasm+
to compile; see the warnings you get when you compile this fragment.
The warnings come from the fact that in the left block, 
the results inside the if statement are written to (local)
registers defined outside the blocks. This is easily fixed in the right hand
side block by writing the results back into a MemValue variable.
